\chapter{Overview and Introduction}

%---------------------------------------------------------------------------------------------------
\section{acknowledgements}

It is my pleasure to express appreciation to people who have contributed to this effort, and without
whom, \bmad would only be a shadow of what it is today: To David Rubin for his support all these
years, to \'Etienne Forest (aka Patrice Nishikawa) for use of his remarkable PTC/FPP library (not to
mention his patience in explaining everything to me), to Desmond Barber for very useful discussions
on how to simulate spin, to Mark Palmer, Matt Rendina, and Attilio De~Falco for all their work
maintaining the build system and for porting \bmad to different platforms, to Frank Schmidt and CERN
for permission to use the \mad tracking code. To Hans Grote and CERN for granting permission to
adapt figures from the \mad manual for use in this one, to Martin Berz for his DA package, and to
Dan Abell, Jacob Asimow, Ivan Bazarov, Moritz Beckmann, Scott Berg, Oleksii Beznosov, Joel Brock,
Sarah Buchan, Avishek Chatterjee, Jing Yee Chee, Christie Chiu, Joseph Choi, Robert Cope, Jim
Crittenden, Laurent Deniau, Gerry Dugan, Michael Ehrlichman, Jim Ellison, Ken Finkelstein, Mike
Forster, Thomas Gl{\"a}{\ss}le, Juan Pablo Gonzalez-Aguilera, Colwyn Gulliford, Klaus Heinemann,
Richard Helms, Georg Hoffstaetter, Henry Lovelace III, Chris Mayes, Karthik Narayan, Katsunobu Oide,
Tia Plautz, Matt Randazzo, Michael Saelim, Jim Shanks, Matthew Signorelli, Hugo Slepicka, Jeff
Smith, Jeremy Urban, Ningdong Wang, Suntao Wang, Mark Woodley, and Demin Zhou for their help.

%---------------------------------------------------------------------------------------------------
\section{What is Bmad?}

The original Bmad was developed as a subroutine
library ("toolkit") for charged--particle and X-Ray simulations in accelerators and storage rings and 
has served as the calculational engine
for many accelerator simulation programs including the \tao program which is widely used in the
accelerator community. \bmad has
been developed at the Cornell Laboratory for Accelerator-based ScienceS and Education (CLASSE) and
has been in use since 1996.
Eventually, in 2023, the organic growth of Bmad --- leading to the code not being as well structured
as it should be --- pushed the decision that the code needed to be
refactored. On top of this was the realization that, while the modern Fortran object-orientated
language that \bmad was written in was a reasonable choice from a purely technical standpoint, 
the Fortran community had atrophied to the point where Fortran compiler maintenance was severely 
affected. The choice was made to use the \julia language for the rewrite.

The name ``\bmad" thus has several meanings. Originally, the term only referred to the \bmad
toolkit. As time went on, and \bmad was used in more and more programs, the term \bmad was
applied to mean the whole ecosystem of toolkit plus programs. Finally, there is refactored
\bmad. This \bmad is a \julia package and as such has is more akin to \bmad-the-ecosystem as
opposed to \bmad-the-toolkit in that \bmad-the-package is used both for constructing lattices
(like \bmad-the-toolkit) and for simulation work without an intermediate program interfacing in between.

%---------------------------------------------------------------------------------------------------
\section{History}

\bmad (Otherwise known as ``Baby MAD" or ``Better MAD" or just plain ``Be MAD!") is a subroutine
library for charged--particle and X-Ray simulations in accelerators and storage rings. \bmad has
been developed at the Cornell Laboratory for Accelerator-based ScienceS and Education (CLASSE) and
has been in use since 1996.

Prior to the development of \bmad, simulation programs at Cornell were written almost from scratch
to perform calculations that were beyond the capability of existing, generally available software.
This practice was inefficient, leading to much duplication of effort. Since the development of
simulation programs was time consuming, needed calculations where not being done. As a response, the
\bmad subroutine library, using an object oriented approach and written in modern object-oriented
Fortran, were developed. The aim of the \bmad project was to:
\begin{Itemize}
\item Cut down on the time needed to develop programs.
\item Cut down on programming errors.
\item Provide a simple mechanism for lattice function calculations
from within control system programs.
\item Provide a flexible and powerful lattice input format.
\item Standardize sharing of lattice information between 
programs.
\end{Itemize}

\bmad can be used to study both single and multi--particle beam dynamics as well as X-rays.  Over
the years, \bmad modules have been developed for simulating a wide variety of phenomena including
intra beam scattering (IBS), coherent synchrotron radiation (CSR), Wakefields, Touschek scattering,
higher order mode (HOM) resonances, etc., etc.  \bmad has various tracking algorithms including
Runge--Kutta and symplectic (Lie algebraic) integration. Wakefields, and radiation excitation and
damping can be simulated. \bmad has routines for calculating transfer matrices, emittances, Twiss
parameters, dispersion, coupling, etc. The elements that \bmad knows about include quadrupoles, RF
cavities (both storage ring and LINAC accelerating types), solenoids, dipole bends, Bragg crystals
etc.  In addition, elements can be defined to control the attributes of other elements. This can be
used to simulate the ``girder'' which physically support components in the accelerator or to easily
simulate the action of control room ``knobs'' that gang together, say, the current going through a
set of quadrupoles.

%---------------------------------------------------------------------------------------------------
\section{Why Julia?}

The choice of \julia as the basis for the new \bmad was not an easy one. Other possibilities included
C++ \cite{}, Python \cite{}, a combination of Python and C/C++, etc. If the \bmad refactoring
project had been started before 2023 the choice probably would have been Python/C/C++. But in
2023 \julia development was mature enough, and the advantages of \julia over the alternatives was
large enough, so that the decision was made to use \julia.

* Short history of \julia

* \julia was constructed for simulations/large data handling.

* Very active community (not Fortran)

But what is the compelling reason for using \julia? First of all, \julia is a scripting language which
means that it is 
like Python. 

%---------------------------------------------------------------------------------------------------
\section{Manual Organization}

As a consequence of \bmad being a software library, this manual serves two masters: The
programmer who wants to develop applications and needs to know about the inner workings of
\bmad, and the user who simply needs to know about the \bmad standard input format and
about the physics behind the various calculations that \bmad performs.

\index{MAD|hyperbf}
To this end, this manual is divided into three parts. The first two
parts are for both the user and programmer while the third part is
meant just for programmers. 
  \begin{description}
  \item[Part~I] \Newline
Part~I discusses the \bmad lattice input standard. The \bmad lattice input standard was
developed using the \mad\cite{b:maduser,b:madphysics}. lattice input standard as a
starting point but, as \bmad evolved, \bmad's syntax has evolved with it.
  \item[Part~II] \Newline
part~II gives the conventions used by \bmad --- coordinate systems, magnetic field
expansions, etc. --- along with some of the physics behind the calculations. By necessity,
the physics documentation is brief and the reader is assumed to be familiar with high
energy accelerator physics formalism.
  \item[Part~III] \Newline
Part~III gives the nitty--gritty details of the \bmad
subroutines and the structures upon which they are based.
\end{description}

\index{Bmad!information}
More information, including the most up--to--date version of this manual, can be found at
the \bmad web site\cite{b:bmad.web}.  Errors and omissions are a fact of life for any
reference work and comments from you, dear reader, are therefore most welcome. Please send
any missives (or chocolates, or any other kind of sustenance) to:
\begin{example}
  David Sagan <dcs16@cornell.edu>
\end{example}
\index{Bmad!error reporting}

The \bmad manual is organized as reference guide and so does not do a good job of instructing the
beginner as to how to use \bmad. For that there is an introduction and tutorial on \bmad and \tao
(\sref{s:tao.intro}) concepts that can be downloaded from the \bmad web page. Go to either the \bmad or
\tao manual pages and there will be a link for the tutorial.

