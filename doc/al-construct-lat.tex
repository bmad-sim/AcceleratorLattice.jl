\chapter{Constructing a Lattice}
\label{c:construct-lat}

%---------------------------------------------------------------------------------------------------
\section{Switches}
\label{s:switch}

A \vn{switch} is a switch category identifier name, conventionally with the word \vn{Switch} at the end, 
with a set of possible values. Switches are like enums without the associated integer. 
The advantage of switches is that a given switch value name can be used for different switches. 
Additionally, switches provide for a bit cleaner code.

For example, 

How to create switches...





%---------------------------------------------------------------------------------------------------
\section{Defining an Element}
\label{s:ele.def}

The \julia language itself is used as the basis for constructing lattices. Other simulation programs
have similarly utilized the underlying programming language for constructing lattices\cite{merlin++,xsuite},
but this is in marked contrast to such programs as MAD\cite{mad}, Elegant\cite{elegant}, and the 
original \bmad\cite{bmad-orig}. 

Chapter~\sref{c:ele} gives a list of elements defined by \Bmad. Elements are defined using the \vn{@ele}
macro. The general syntax is:
\begin{example}
  @ele eleName = eleType(param1 = val1, param2 = val2, ...)
\end{example}
where \vn{eleName} is the name of the element, \vn{eleType} is the type of element, \vn{param1}, \vn{param2},
etc. are parameter names and \vn{val1}, \vn{val2}, etc. are the parameter values.
Example:
\begin{example}
  @ele qf = Quadrupole(len = 0.6, K1 = 0.370)
\end{example}
The \vn{@ele} macro will construct a \julia variable with the name \vn{eleName}. Additionally the element
that this variable references will also hold \vn{eleName} as the name of the element.

To copy an element use the \vn{deepcopy} constructor.


