\chapter{Constructing Lattices}
\label{c:construct-lat}


%---------------------------------------------------------------------------------------------------
\section{Defining a Lattice Element}
\label{s:ele.def}


Chapter~\sref{c:ele} gives a list of lattice elements defined by \accellat. 
Lattice elements are instantiated from structs which inherit from the abstract type \vn{Lat}.

Elements are defined using the \vn{@ele} macro. The general syntax is:
\begin{example}
  @ele eleName = eleType(param1 = val1, param2 = val2, ...)
\end{example}
where \vn{eleName} is the name of the element, \vn{eleType} is the type of element, \vn{param1}, \vn{param2},
etc. are parameter names and \vn{val1}, \vn{val2}, etc. are the parameter values.
Example:
\begin{example}
  @ele qf = Quadrupole(L = 0.6, K1 = 0.370)
\end{example}
The \vn{@ele} macro will construct a \julia variable with the name \vn{eleName}. Additionally the element
that this variable references will also hold \vn{eleName} as the name of the element. So with this
example, \vn{qf.name} will be the string \vn{"qf"}. If multiple elements are being defined, a single
\vn{@eles} macro can be used instead of multiple \vn{@ele} macros. Example:
\begin{example}
  @eles begin
    s1 = Sextupole(L = ...)
    b2 = Bend(...)
    ...
  end
\end{example}

%---------------------------------------------------------------------------------------------------
\subsection{Anatomy of an Element}
\label{s:ele.type}

The structs for all elements types contain exactly one component which is a Dict called \vn{pdict}
(short for ``parameter dict''). With 





To copy an element use the \vn{deepcopy} constructor.

%---------------------------------------------------------------------------------------------------
\section{Defining a Lattice Element Type}
\label{s:ele.type}

All lattice element types like \vn{Quadrupole}, \vn{Marker}, etc. are subtypes of the abstract type
\vn{Ele}. To construct a new type, use the \vn{\@construct_ele_type} macro. Example:
\begin{example}
  @construct_ele_type MyEleType
\end{example}

%---------------------------------------------------------------------------------------------------
\section{Lattice Element Internals}
\label{s:ele.inside}

All element types have a single component called \vn{pdict} (``parameter dict'') which is of
type \vn{Dict\{Symbol,Any\}}. Using a \vn{Dict} has advantages and disadvantages. The advantage is
that an element is not restricted as to what can be stored in it. The disadvantage is that it is not
type stable (\sref{s:type.stable}). This is generally acceptable when lattices are constructed but
is undesirable during tracking. To regain type stability during tracking, element parameters are
put into immutable structs called \vn{element parameter} groups 
and these structs are stored in \vn{pdict}. During tracking, the tracking
code can access element parameters via the struct which makes the code type stable as will be illustrated below.

The \vn{element parameter} group structures are all subtypes of the abstract type \vn{EleParameterGroup}.
For example, the \vn{LengthGroup} holds the length and s-positions of the element:
\begin{example}
  @kwdef struct LengthGroup <: EleParameterGroup
    L::Float64 = 0
    s::Float64 = 0
    s_downstream::Float64 = 0
  end
\end{example}
The \vn{\@kwdef} macro automatically defines a keyword-based constructor for \vn{LengthGroup}. 
When a parameter group is stored in an element's \vn{pdict}, the key will be the symbol associated
with the struct which in this case is \vn{:LengthGroup}. For example, an element's length can be
accessed via \vn{ele.pdict[:LengthGroup].L}. 




