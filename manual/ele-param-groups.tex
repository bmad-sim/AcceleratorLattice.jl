\chapter{Element Parameter Groups}
\label{c:ele.groups}

Generally, element parameters are grouped into ``\vn{element} \vn{parameter} \vn{group}'' 
types which is discussed in \sref{c:ele}.

Element parameter groups inherit from the abstract type \vn{EleParameterGroup} which
in turn inherits from \vn{BaseEleParameterGroup}. Some
parameter groups have sub-group components. These sub-groups also inherit from \vn{BaseEleParameterGroup}:
\begin{example}
  abstract type BaseEleParameterGroup end
  abstract type EleParameterGroup <: BaseEleParameterGroup end
  abstract type EleParameterSubGroup <: BaseEleParameterGroup end
\end{example}

All parameter groups have associated docstrings that can be accessed when using the REPL.

The parmeters groups are:
\begin{table}[htb]
\centering
{\tt
\begin{tabular}{llll} \toprule
  {\it Group}        & {\it Section}         & {\it Group}      & {\it Section}         \\ \midrule
 AlignmentGroup      & \sref{s:align.g}      & ApertureGroup    & \sref{s:apert.g}      \\
 BMultipoleGroup     & \sref{s:bmult.g}      & BeamBeamGroup    & \sref{s:bb.g}         \\
 BendGroup           & \sref{s:bend.g}       & EMultipoleGroup  & \sref{s:emult.g}      \\
 FloorPositionGroup  & \sref{s:floor.g}      & GirderGroup      & \sref{s:girder.g}     \\
 InitParticleGroup   & \sref{s:initp.g}      & LCavityGroup     & \sref{s:lcav.g}       \\
 LengthGroup         & \sref{s:length.g}     & LordSlaveGroup   & \sref{s:lord.slave.g} \\
 MasterGroup         & \sref{s:master.g}     & PatchGroup       & \sref{s:patch.g}      \\
 RFFieldGroup        & \sref{s:rffield.g}    & RFGroup          & \sref{s:rf.g}         \\
 RFMasterGroup       & \sref{s:rfmaster.g}   & ReferenceGroup   & \sref{s:ref.g}        \\
 SolenoidGroup       & \sref{s:sol.g}        & StringGroup      & \sref{s:string.g}     \\
 TrackingGroup       & \sref{s:track.g}      & TwissGroup       & \sref{s:twiss.g}      \\
  \bottomrule
\end{tabular}
} 
\caption{Table of element parameter groups.}
\label{t:particle.groups}
\end{table}

* Note: NaN denotes parameter that is not set.

* beta_ref is a dependent attribute.

%---------------------------------------------------------------------------------------------------
\section{Enums}

\al uses the package \vn{EnumX.jl} to define enums (enumerated numbers).
The \vn{\@enumit} macro is used to convert a list into an enum group and export the names.
\vn\@enumit} also overloads \vn{Base.string} so that something like \vn{string(Lord.NOT)} will 
return \vn{"Lord.NOT"} instead of just \vn{"NOT"}.

The following enum groups are defined:

%---------------------------------------------------------------------------------------------------
\section{AlignmentGroup}
\label{s:align.g}

The alignment group give the orientation of an element (\sref{s:orient}). Specifically, the
orientation of the elements body coordinates with respect to the machine coordinates.