\chapter{Element Parameter Groups}
\label{c:ele.groups}

Generally, element parameters are grouped into ``\vn{element} \vn{parameter} \vn{group}'' 
types which is discussed in \sref{c:ele}.

Element parameter groups inherit from the abstract type \vn{EleParameterGroup} which
in turn inherits from \vn{BaseEleParameterGroup}. Some
parameter groups have sub-group components. These sub-groups also inherit from \vn{BaseEleParameterGroup}:
\begin{example}
  abstract type BaseEleParameterGroup end
  abstract type EleParameterGroup <: BaseEleParameterGroup end
  abstract type EleParameterSubGroup <: BaseEleParameterGroup end
\end{example}

All parameter groups have associated docstrings that can be accessed when using the REPL.

The parmeters groups are:
\begin{table}[htb]
\centering
{\tt
\begin{tabular}{llll} \toprule
  {\it Group}        & {\it Section}         & {\it Group}      & {\it Section}         \\ \midrule
 AlignmentGroup      & \sref{s:align.g}      & ApertureGroup    & \sref{s:apert.g}      \\
 BMultipoleGroup     & \sref{s:bmult.g}      & BeamBeamGroup    & \sref{s:bb.g}         \\
 BendGroup           & \sref{s:bend.g}       & EMultipoleGroup  & \sref{s:emult.g}      \\
 FloorPositionGroup  & \sref{s:floor.g}      & GirderGroup      & \sref{s:girder.g}     \\
 InitParticleGroup   & \sref{s:initp.g}      & LCavityGroup     & \sref{s:lcav.g}       \\
 LengthGroup         & \sref{s:length.g}     & LordSlaveGroup   & \sref{s:lord.slave.g} \\
 MasterGroup         & \sref{s:master.g}     & PatchGroup       & \sref{s:patch.g}      \\
 RFFieldGroup        & \sref{s:rffield.g}    & RFGroup          & \sref{s:rf.g}         \\
 RFMasterGroup       & \sref{s:rfmaster.g}   & ReferenceGroup   & \sref{s:ref.g}        \\
 SolenoidGroup       & \sref{s:sol.g}        & StringGroup      & \sref{s:string.g}     \\
 TrackingGroup       & \sref{s:track.g}      & TwissGroup       & \sref{s:twiss.g}      \\
  \bottomrule
\end{tabular}
} 
\caption{Table of element parameter groups.}
\label{t:particle.groups}
\end{table}

* Note: NaN denotes parameter that is not set.

* beta_ref is a dependent attribute.

%---------------------------------------------------------------------------------------------------
\section{Enums}
\label{s:enums}

\accellat uses the package \vn{EnumX.jl} to define enums (enumerated numbers).
Essentially what happens is that for each enum group there is a group name, For example \vn{BendType},
along with a set of values which, for \vn{BendType}, is \vn{SECTOR} and \vn{RECTANGULAR}. Values
are always referred to by their "full" name which in this example is \vn{BendType.SECTOR} and
\vn{BendType.RECTANGULAR}. Exception: \vn{BranchGeometry.CLOSED} and \vn{BranchGeometry.OPEN} are
used often enough so that the constants \vn{OPEN} and \vn{CLOSED} are defined.

The group name followed by a \vn{.T} suffix denotes the enum type.
For example:
\begin{example}
  struct ApertureGroup <: EleParameterGroup
    aperture_type::ApertureShape.T = ApertureShape.ELLIPTICAL
    aperture_at::BodyLoc.T = BodyLoc.ENTRANCE_END
    ...
\end{example}

The \vn{enumit} function is used to convert a list into an enum group and export the names.
The \vn{enumit} function also overloads \vn{Base.string} so that something like \vn{string(Lord.NOT)} will 
return \vn{"Lord.NOT"} instead of just \vn{"NOT"} (an issue with the EnumX.jl package). See the
documentation for \vn{enumit} for more details.

The \vn{enum_add} function is used to add values to an existing enum group. See the documentation for
\vn{enum_add} for more details. This function is used with code extensions to customize \accellat.


Below is a list of enum groups defined in \accellat. 
\begin{description}
%
\item[ApertureShape] --- The shape of an aperture.\Newline 
\hspace*{-20pt}
\begin{tabular}{ll}
  RECTANGULAR & --- Rectangular shape \\
  ELLIPTICAL  & --- Elliptical shape \\
\end{tabular}
%
\item[BendType] --- Type of Bend element magnet.\Newline
\hspace*{-20pt}
\begin{tabular}{ll}
  SECTOR      & --- Sector shape\\
  RECTANGULAR & --- Rectangular shape \\
\end{tabular}
%
\item[BodyLoc] --- Longitudinal location with respect to element's body coordinates.\Newline
\hspace*{-20pt}
\begin{tabular}{ll}
  ENTRANCE_END & --- Body entrance end \\
  CENTER       & --- Element center \\
  EXIT_END     & --- Body exit end \\
  BOTH_ENDS    & --- Both ends \\
  NOWHERE      & --- No location \\
  EVERYWHERE   & --- Everywhere \\
\end{tabular}
%
\item[BranchGeometry] --- Geometry of a lattice branch\Newline
\hspace*{-20pt}
\begin{tabular}{ll}
  OPEN    & --- Open geometry like a Linac. \\
  CLOSED  & --- Closed geometry like a storage ring.
\end{tabular}
%
\item[Cavity] --- Type of RF cavity. \Newline
\hspace*{-20pt}
\begin{tabular}{ll}
  STANDING_WAVE   & --- Standing wave cavity \\
  TRAVELING_WAVE  & --- Traveling wave cavity \\
\end{tabular}
%
\item[Lord] --- Type of lord an element is \Newline
\hspace*{-20pt}
\begin{tabular}{ll}
  NOT       & --- Not a lord \\
  SUPER     & --- Super lord \\
  MULTIPASS & --- Multipass lord \\
  GOVERNOR  & --- Girder and other "minor" lords \\ 
\end{tabular}
%
\item[Slave] --- Type of slave an element is \Newline
\hspace*{-20pt}
\begin{tabular}{ll}
  NOT       & --- Not a slave \\
  SUPER     & --- Super slave \\
  MULTIPASS & --- Multipass slave \\
\end{tabular}
%
\item[Loc] --- Longitudinal location with respect to element's machine coordinates. \Newline
\hspace*{-20pt}
\begin{tabular}{ll}
  UPSTREAM_END   & --- Upstream element end\\
  CENTER         & --- center of element \\
  INSIDE         & --- Somewhere inside \\
  DOWNSTREAM_END & --- Downstream element end \\
\end{tabular}
%
\item[Select] --- Specifies where to select if there is a choice of elements or positions. \Newline
\hspace*{-20pt}
\begin{tabular}{ll}
  UPSTREAM   & --- Select upstream \\
  DOWNSTREAM & --- Select downstream \\
\end{tabular}
%
\item[ExactMultipoles] --- How multipoles are handled in a Bend element \Newline
\hspace*{-20pt}
\begin{tabular}{ll}
  OFF               & --- Bend curvature not taken into account. \\
  HORIZONTALLY_PURE & --- Multipole coefficients interpreted being as being horizontally pure. \\
  VERTICALLY_PURE   & --- Multipole coefficients interpreted being as being vertically pure. \\
\end{tabular}
%
\item[FiducialPt] Fiducial point location with respect to element's machine coordinates. \Newline
\hspace*{-20pt}
\begin{tabular}{ll}
  ENTRANCE_END & --- Entrance end of element \\
  CENTER       & --- Center of element \\
  EXIT_END     & --- Exit end of element \\
  NONE         & --- No point chosen \\
\end{tabular}
%
\end{description}

%---------------------------------------------------------------------------------------------------
\section{AlignmentGroup}
\label{s:align.g}

The alignment group give the orientation of an element (\sref{s:orient}). Specifically, the
orientation of the elements body coordinates with respect to the machine coordinates.