\chapter{Lattice Elements}
\label{c:elements}

%---------------------------------------------------------------------------------------------------

A lattice is made up of a collection of elements --- quadrupoles,
bends, etc. This chapter discusses the various types of elements
available in \bmad.

\begin{table}[htb]
\centering
{\tt
\begin{tabular}{llll} \toprule
  {\it Element}    & {\it Section}         & {\it Element}      & {\it Section}       \\ \midrule
  BeamBeam         & \ref{s:beambeam}      &  Marker            & \ref{s:mark}        \\ 
  BeginningEle     & \ref{s:begin.ele}     &  Mask              & \ref{s:mask}        \\
  Bend             & \ref{s:bend}          &  Multipole         & \ref{s:mult}        \\
  Collimator       & \ref{s:col}           &  NullEle           & \ref{s:null.ele}    \\
  Converter        & \ref{s:converter}     &  Octupole          & \ref{s:oct}         \\
  CrabCavity       & \ref{s:crab}          &  Patch             & \ref{s:patch}       \\
  Custom           & \ref{s:custom}        &  Pipe              & \ref{s:monitor}     \\  
  Drift            & \ref{s:drift}         &  Quadrupole        & \ref{s:quad}        \\
  EGun             & \ref{s:e.gun}         &  RFbend            & \ref{s:rf.bend}     \\
  ElSeparator      & \ref{s:elsep}         &  RFcavity          & \ref{s:rfcav}       \\ 
  EMField          & \ref{s:em.field}      &  SadMult           & \ref{s:sad.mult}    \\
  Fiducial         & \ref{s:fiducial}      &  Sextupole         & \ref{s:sex}         \\
  FloorShift       & \ref{s:floor.ele}     &  Solenoid          & \ref{s:sol}         \\
  Foil             & \ref{s:foil}          &  Taylor            & \ref{s:taylor}      \\
  Fork             & \ref{s:fork}          &  ThickMultipole    & \ref{s:thick.mult}  \\
  Instrument       & \ref{s:monitor}       &  Undulator         & \ref{s:wiggler}     \\
  Kicker           & \ref{s:kicker}        &  UnionEle          & \ref{s:union.ele}   \\
  Lcavity          & \ref{s:lcav}          &  Wiggler           & \ref{s:wiggler}     \\
  \bottomrule
\end{tabular}
} \caption{Table of element types suitable for use with charged particles. Also see
Table~\ref{t:control.classes}} \label{t:particle.classes}
\end{table}

The list of element types known to \bmad is shown
in Table~\ref{t:particle.classes}, \ref{t:photon.classes}, and \ref{t:control.classes}.
Table~\ref{t:particle.classes} lists the elements suitable for use with charged particles,
Table~\ref{t:photon.classes} which lists the elements suitable for use with photons, and finally
Table~\ref{t:control.classes} lists the \vn{controller} element types that can be used for parameter
control of other elements. Note that some element types are suitable for both particle and photon
use.

\begin{table}[ht]
\centering
{\tt
\begin{tabular}{llll} \toprule
  {\it Element}      & {\it Section}         & {\it Element}         & {\it Section}       \\ \midrule
  Beginning_Ele      & \ref{s:begin.ele}     &    Lens               & \ref{s:lens}        \\
  Capillary          & \ref{s:capillary}     &  Marker               & \ref{s:mark}        \\
  Crystal            & \ref{s:crystal}       &  Mask                 & \ref{s:mask}        \\
  Custom             & \ref{s:custom}        &  Match                & \ref{s:match}       \\
  Detector           & \ref{s:detector}      &  Monitor              & \ref{s:monitor}     \\ 
  Diffraction_Plate  & \ref{s:diff.plate}    &  Mirror               & \ref{s:mirror}      \\
  Drift              & \ref{s:drift}         &  Multilayer_Mirror    & \ref{s:multilayer}  \\
  Ecollimator        & \ref{s:col}           &  Patch                & \ref{s:patch}       \\
  Fiducial           & \ref{s:fiducial}      &  Photon_Fork          & \ref{s:fork}        \\
  Floor_Shift        & \ref{s:floor.ele}     &  Photon_Init          & \ref{s:photon.init} \\
  Fork               & \ref{s:fork}          &  Pipe                 & \ref{s:monitor}     \\
  GKicker            & \ref{s:gkicker}       &  Rcollimator          & \ref{s:col}         \\
  Instrument         & \ref{s:monitor}       &  Sample               & \ref{s:sample}      \\
  \bottomrule
\end{tabular}
}
\caption{Table of element types suitable for use with photons. Also see Table~\ref{t:control.classes}}
\label{t:photon.classes}
\end{table}

\begin{table}[ht]
\centering
{\tt
\begin{tabular}{llll} \toprule
  {\it Element}  & {\it Section}     & {\it Element}  & {\it Section}    \\ \midrule
  Controller     & \ref{s:group}     &  Ramper        & \ref{s:ramper}   \\
  Girder         & \ref{s:girder}    &                &                  \\
 \\ \bottomrule
\end{tabular}
}
\caption{Table of controller elements.}
\label{t:control.classes}
\end{table}

For a listing of element attributes for each type of element, see Chapter~\sref{c:attrib.list}.

\newpage

%---------------------------------------------------------------------------------------------------
\section{Lattice Element Parameters}

Before discussing lattice elements themselves, the element parameters need to be discussed first.
Element parameters are divided into immutable struct groups which inherit from the abstract type
\vn{EleParameterGroup}. A list of parameter groups can be seen using the command

For example, the position of the element with respect

Element parameters are listed in 

%---------------------------------------------------------------------------------------------------
\section{Anatomy of a Lattice Element}

All lattice elements inherit from the abstract type \vn{Ele}. There is a macro \vn{construct_ele_type}
that is used to construct a new type of element. For example:
\begin{example}
  @construct_ele_type Bend
\end{example}
this defines the immutable \vn{Bend} struct which inherits from \vn{Ele}. 

All element structs have a single \vn{Dict\{Symbol,Any\}} field called \vn{param}.
The dot selection operator has been overloaded so that something like \vn{ele.name}
is mapped to \vn{ele.param[:name]}. Except!


