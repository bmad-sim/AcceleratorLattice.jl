\chapter{Lattice Elements}
\label{c:ele}

This chapter discusses lattice elements including how to create them and how to manipulate them.
For a list of lattice elements defined by \accellat, see chapter~\sref{c:ele.types}. 

%---------------------------------------------------------------------------------------------------
\section{Instantiating a Lattice Element}
\label{s:ele.def}

Elements are defined using the \vn{@ele} or \vn{@eles} macros. 
The general syntax of the \vn{@ele} macro is:
\begin{example}
  @ele eleName = eleType(param1 = val1, param2 = val2, ...)
\end{example}
where \vn{eleName} is the name of the element, \vn{eleType} is the type of element, 
\vn{param1}, \vn{param2},
etc. are parameter names and \vn{val1}, \vn{val2}, etc. are the parameter values.
Example:
\begin{example}
  @ele qf = Quadrupole(L = 0.6, K1 = 0.370)
\end{example}
The \vn{@ele} macro will construct a \julia variable with the name \vn{eleName}. 
Additionally, the element
that this variable references will also hold \vn{eleName} as the name of the element. So with this
example, \vn{qf.name} will be the string \vn{"qf"}. If multiple elements are being defined in a 
group, a single
\vn{@eles} macro can be used instead of multiple \vn{@ele} macros using the syntax:
\begin{example}
  @eles begin
    eleName1 = eleType1(p11 = v11, p12 = v12, ...)
    eleName2 = eleType2(p21 = v21, p22 = v22, ...)
    ... etc...
  end
\end{example}
Example:
\begin{example}
  @eles begin
    s1 = Sextupole(L = ...)
    b2 = Bend(...)
    ...
  end
\end{example}

%---------------------------------------------------------------------------------------------------
\section{Element Parameters}
\label{s:ele.type}

Generally, element parameters are grouped into ``\vn{element} \vn{parameter} \vn{group}'' 
types which inherit from the abstract type \vn{EleParameterGroup}.
For example, the \vn{LengthGroup} holds the length and s-positions of the element and is defined by:
\begin{example}
  @kwdef struct LengthGroup <: EleParameterGroup
    L::Number = 0.0               # Length of element
    s::Number = 0.0               # Starting s-position
    s_downstream::Number = 0.0    # Ending s-position
    orientation::Int = 1          # Longitudinal orientation
  end
\end{example}
The \vn{@kwdef} macro automatically defines a keyword-based constructor for \vn{LengthGroup}.
See the Julia manual for more information on \vn{@kwdef}. 
To see a list of all element parameter groups use the \vn{subtypes(EleParamterGroup)} command.
To see the components of a given group use the \vn{fieldnames} function. For information on
a given element parameter use the \vn{info(::Symbol)} function. For example, the information on
the \vn{s_downstream} parameter which is a field of the \vn{LengthGroup} is:
\begin{example}
  julia> info(:s_downstream)
    User name:       s_downstream
    Stored in:       LengthGroup.s_downstream
    Parameter type:  Number
    Units:           m
    Description:     Longitudinal s-position at the downstream end.
\end{example}
Notice that the argument to the \vn{info} function is the symbol associated with the parameter.
the ``user name'' is the name used when setting the parameter. This is discussed below. For most
parameters, the user name and the name of the corresponding component in the element parameter
group is the same. But there are exceptions. For example:
\begin{example}
  julia> info(:theta)
    User name:       theta_floor
    Stored in:       FloorPositionGroup.theta
    Parameter type:  Number
    Units:           rad
    Description:     Element floor theta angle orientation
\end{example}
In this example, the user name is \vn{theta_floor} so that this parameter can be set via
\begin{example}
  @ele bg = BeginningEle(theta_floor = 0.3)    # Set at element definition time.
  bg.theta_floor = 2.7                         # Or set after definition.
\end{example}
But the component in the \vn{FloorPositionGroup} is \vn{theta} so
\begin{example}
  bg.FloorPositionGroup.theta = 2.7   # Equivalent to the set above.
\end{example}

%---------------------------------------------------------------------------------------------------
\section{Anatomy of an Element}
\label{s:ele.anatomy}

Lattice element types (\vn{Quadrupole}, \vn{RFCavity}, etc.) inherit from the abstract 
type \vn{Ele}. To see a list of all element types, use the command \vn{subtypes(Ele)}:
\begin{example}
  julia> subtypes(Ele)
  41-element Vector{Any}:
   ACKicker
   BeamBeam
   BeginningEle
   Bend
   ...
\end{example}

All lattice element types have a single field which is a Dict named \vn{pdict} (parameter dict)
of type \vn{Dict\{Symbol,Any\}}. 
The values of \vn{pdict} will, with a few exceptions, be an
element parameter group. The corresponding key for a parameter group in \vn{pdict} is the symbol associated 
with the type. For example, a \vn{LengthGroup} would be stored in \vn{pdict[:LengthGroup]}.

To hide the complexity of parameter groups, the dot selection operator is overloaded for elements.
For example, if \vn{q} is an element instance, \vn{q.pdict[:LengthGroup].s} can be accessed
via \vn{q.s}.

The \vn{Base.setproperty} and \vn{Base.getproperty} functions, which get called when the dot
selection operator is used, have been overloaded for elements so that \vn{ele.L} will get mapped to 
\vn{ele.pdict[:LengthGroup].L}. Thus the following two statements both set the \vn{s_downstream}
parameter of an element named \vn{q1}:
\begin{example}
  q1.pdict[:Length_group].s_downstream = q1.pdict[:Length_group].s + 
                                                     q1.pdict[:Length_group].L
  q1.s_downstream = q1.s + q1.L
\end{example}
These two statements are not equivalent however. The difference is that when \vn{s_downstream}
is set using \vn{q1.s_downstream}, the set is recorded by adding an entry to a 
Dict in the element at \vn{ele.pdict[:changed]} where \vn{ele} is the element whose parameter
was changed. The key of the entry will be the symbol (\vn{:s_downstream} in the present example)
associated with the parameter and the value will be the old value of the parameter. When
the \vn{bookkeeper(::Lat)} function is called, the bookkeeping code will use the
entries in \vn{ele.pdict[:changed]} to limit the bookkeeping to what is necessary and thus
minimize computing time. Knowing what has been changed is also important in resolving what
parameters need to be changed. 
For example, if the bend \vn{angle} of a bend is changed, the bookkeeping code will set the 
bending strength \vn{g} using the equation \vn{g} = \vn{angle} / \vn{L}. If, instead, if
\vn{g} is changed, the bookkeeping code will set \vn{angle} appropriately. 



%---------------------------------------------------------------------------------------------------
\section{Bookkeeping and Dependent Element Parameters}
\label{s:param.depend}

After lattice parameters are changed, the function \vn{bookkeeper(::Lat)} needs to be called
so that dependent parameters can be updated. 
Since bookkeeping can take a significant amount of time if bookkeeping is done every time
a change to the lattice is made, and since there is no good way to tell when bookkeeping should
be done, After lattice expansion, \vn{bookkeeper(::Lat)} is never called directly by \accellat 
functions and needs to be called by the User when appropriate (generally before tracking or
other computations are done).

Broadly there are two types of dependent parameters: intra-element dependent parameters where
the changed parameters and the dependent parameters are all within the same element and
cascading dependent parameters where changes to one element cause changes to parameters of 
elements downstream.

The cascading dependencies are:
\begin{description}
%
\item [s-position dependency:]
Changes to an elements length \vn{L} or changes to the beginning element's \vn{s} parameter will
result in the s-positions of all downstream elements changing.
%
\item [Reference energy dependency:] Changes to the be beginning element's reference energy (or
equivilantly the referece momentum), or changes to the \vn{voltage} of an \vn{LCavity} element
will result in the reference energy of all downstream elements changing.
%
\item[Global position dependency:]
The position of a lattice element in the global coordinate system (\sref{s:global}) is affected
by a) the lengths of all upstream elements, b) the bend angles of all upstream elements, and c)
the position in global coordinates of the beginning element.
\end{description}


%---------------------------------------------------------------------------------------------------
\section{Defining a New Element Type}
\label{s:ele.type}

To construct a new type, use the \vn{@construct_ele_type} macro. Example:
\begin{example}
  @construct_ele_type MyEle
\end{example}
And this defines a new type called \vn{MyEle} which inherits from the abstract type \vn{Ele} and
defines \vn{MyEle} to have a single field called \vn{pdict} which is of type \vn{Dict\{Symbol,Any\}}.
This macro also pushes the name
