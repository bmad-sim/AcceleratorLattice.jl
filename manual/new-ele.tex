\chapter{Defining New Lattice Elements}
\label{c:new.ele}

%---------------------------------------------------------------------------------------------------
\section{Defining new Element Parameters}

* Bookkeeping 

%---------------------------------------------------------------------------------------------------
\section{Defining a New Element}
\label{s:new.ele}

To construct a new element type:

* Define a new element type. Example:
\begin{example}
  \@construct_ele_type NewEleType
\end{example}

* Extend EleGeometry Holy trait group (\sref{s:holy}) if a new geometry is needed. Example:
\begin{example}
  abstract type CORKSCREW <: EleGeometry end
\end{example}

* If the geometry is not \vn{STRAIGHT}, Extend the \vn{ele_geometry()} function to return the 
correct geometry for the new type of element. Example:
\begin{example}
  ele_geometry(ele::NewEleType) = CORKSCREW
\end{example}

* If the element has a new type of geometry, extend the \vn{propagate_ele_geometry()} function
to handle the new type of geometry. Example:
\begin{example}
  function propagate_ele_geometry(::Type{CORKSCREW}, fstart::OrientationGroup, ele::Ele)
    ...
    return floor_end  # OrientationGroup at the downstream end of the element.
  end
\end{example}


%---------------------------------------------------------------------------------------------------

