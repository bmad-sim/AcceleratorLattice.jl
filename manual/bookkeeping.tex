\chapter{AcceleratorLattice Bookkeeping}
\label{c:bookkeeping}

Bookkeeping in \accellat mainly involves making sure that dependent parameters are updated as needed.

%---------------------------------------------------------------------------------------------------
\section{Element Parameter Access}
\label{s:access}

%---------------------------------------------------------------------------------------------------
\section{Changed Parameters and Auto-Bookkeeping}
\label{s:changed.param}

Importance of using pop!, insert!, push! and set! when modifying the branch.ele array.

The \vn{ele.changed} parameter (which is actually \vn{ele.pdict[:changed]}) is a dictionary.
The keys of this dict will be either symbols of the changed parameters or
will be an element parameter group. 
When the key is a symbol of a changed parameter,
the dict value will be the old value of the parameter. These dict entries are set by the 
overloaded \vn{Base.setproperty(ele, param_sym, value)} function. 
When the key is an element parameter group, the dict value will be the string \vn{"changed"}.
These dict entries are set by functions that do lord/slave bookkeeping.

When bookkeeping is done, entries from the \vn{ele.changed} dict are removed when the corresponding
parameter(s) are bookkeeped. If there are dict entries that remain after all bookkeeping is done,
this is an indication of a problem and a warning message is printed.

