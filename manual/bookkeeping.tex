\chapter{Lattice Bookkeeping}
\label{c:bookkeeping}

Bookkeeping in \accellat mainly involves making sure that dependent parameters are updated as needed.
This includes dependent parameters within a lattice element, propagating changes through the lattice,
and lord/slave bookkeeping.

Note: An element in a branch has a pointer to the branch (\vn{Ele.branch}) and the branch has
a pointer to the lattice (\vn{Branch.lat}). So a lattice element ``knows'' about the lattice it
is in. On the other hand, elements in beam lines don't have pointers to the beamline. This is 
an important factor when setting up forks.

%---------------------------------------------------------------------------------------------------
\section{Lord/Slave Bookkeeping}
\label{s:lord.slave.book}

There are two types of lords/slave groupings:
\begin{example}
  Superposition: Super lords / Super Slaves        \sref{c:super}
  Multipass:     Multipass lords Multipass Slaves  \sref{c:multipass}
\end{example}

The lord and slave slave status of a lattice element is contained in the \vn{LordSlaveStatusParams}
parameter group. The components of this group are (\sref{s:lord.slave.g}):
\begin{example}
  lord_status::Lord.T     - Lord status. 
  slave_status::Slave.T   - Slave status. 
\end{example}

For a given element, some combinations of lord and slave status are not possible. The possibilities are:
\begin{tabular}{lcccc}
  \toprule
  & \multicolumn{4}{c}{\vn{lord_status}} \\
  \cmidrule(lr){2-5}
  {\vn{slave_status}} &
  \begin{sideways}\vn{.NOT}\end{sideways} &
  \begin{sideways}\vn{.SUPER}\end{sideways} &
  \begin{sideways}\vn{.MULTIPASS}\end{sideways} &
  \\ \midrule
  %                   N   S   M
  \vn{.NOT}         & X & X & X \\
  \vn{.SUPER}       & X &   &   \\
  \vn{.MULTIPASS}   & X & X &   \\
  \bottomrule
\end{tabular}
\hfill \break

Notice that the only possibility for an element to simultaneously be both a lord and a slave is
for a super lord being a multipass slave.

%---------------------------------------------------------------------------------------------------
\section{Girders}

\vn{Girders} support a set of supported elements. A \vn{Girder} may support other \vn{Girders}
and so a hierarchy of \vn{Girders} may be constructed. While a \vn{Girder} may support many elements,
any given element may only be supported by one \vn{Girder}.

\vn{Girder} elements may support super and multipass lord elements, a \vn{Girder} will never support
slave elements directly. This includes any super lord element that is also a multipass slave.

A \vn{Girder} element will have a \vn{Vector\{Ele\}} parameter of supported elements \vn{.supported}. 
Supported elements will have a \vn{.girder} parameter pointing to the supporting \vn{Girder}.
Elements that do not have a supporting \vn{Girder} will not have this parameter.

%---------------------------------------------------------------------------------------------------
\section{Superposition}
\label{s:super.book}

Super lords are formed when elements are superimposed on top of other elements (\sref{c:super}).
The \accellat bookkeeping routines and take changes to lord element parameters and set the 
appropriate slave parameters.

When there is a set of lattice elements that are in reality the same physical element, a
multipass lord can be used to represent the common physical element \sref{c:multipass}. 
The \accellat bookkeeping routines and take changes to lord element parameters and set the 
appropriate slave parameters.

\vn{Girder} lords support other elements (possibly including other \vn{Girder} lords). Alignment
shifts of a \vn{Girder} lord will shift the supported elements accordingly.

%---------------------------------------------------------------------------------------------------
\section{Lord/Slave Element Pointers}

All three types of lord elements contain a \vn{Vector\{ele\}} of elements called \vn{slaves}.

%---------------------------------------------------------------------------------------------------
\section{Element Parameter Access}
\label{s:access}

%---------------------------------------------------------------------------------------------------
\section{Changed Parameters and Auto-Bookkeeping}
\label{s:changed.param}

Importance of using pop!, insert!, push! and set! when modifying the branch.ele array.

The \vn{ele.changed} parameter (which is actually \vn{ele.pdict[:changed]}) is a dictionary.
The keys of this dict will be either symbols of the changed parameters or
will be an element parameter group. 
When the key is a symbol of a changed parameter,
the dict value will be the old value of the parameter. These dict entries are set by the 
overloaded \vn{Base.setproperty(ele, param_sym, value)} function. 
When the key is an element parameter group, the dict value will be the string \vn{"changed"}.
These dict entries are set by functions that do lord/slave bookkeeping.

When bookkeeping is done, entries from the \vn{ele.changed} dict are removed when the corresponding
parameter(s) are bookkeeped. If there are dict entries that remain after all bookkeeping is done,
this is an indication of a problem and a warning message is printed.

