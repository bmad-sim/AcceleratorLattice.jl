\chapter{Lattice Element Types}
\label{c:ele.types}

%---------------------------------------------------------------------------------------------------

This chapter discusses the various types of elements
available in \accellat.
Table~\ref{t:particle.classes} lists the elements that can be tracked through.
Table~\ref{t:control.classes} lists the \vn{controller} element types that can be used for parameter
control of other elements. 

\begin{table}[htb]
\centering
{\tt
\begin{tabular}{llll} \toprule
  {\it Element}    & {\it Section}         & {\it Element}      & {\it Section}       \\ \midrule
  BeamBeam         & \ref{s:beambeam}      &  LCavity           & \ref{s:lcav}        \\
  BeginningEle     & \ref{s:begin.ele}     &  Marker            & \ref{s:marker}      \\
  Bend             & \ref{s:bend}          &  Mask              & \ref{s:mask}        \\
  Collimator       & \ref{s:col}           &  Match             & \ref{s:match}       \\ 
  CrabCavity       & \ref{s:crab}          &  NullEle           & \ref{s:nullele}     \\
  Custom           & \ref{s:custom}        &  Octupole          & \ref{s:oct}         \\
  Drift            & \ref{s:drift}         &  Patch             & \ref{s:patch}       \\  
  EGun             & \ref{s:egun}          &  Quadrupole        & \ref{s:quad}        \\ 
  ElSeparator      & \ref{s:elsep}         &  RFCavity          & \ref{s:rfcav}       \\ 
  EMField          & \ref{s:emfield}       &  Sextupole         & \ref{s:sex}         \\ 
  Fiducial         & \ref{s:fiducial}      &  Solenoid          & \ref{s:sol}         \\
  FloorShift       & \ref{s:floorshift}    &  Taylor            & \ref{s:taylor}      \\
  Foil             & \ref{s:foil}          &  ThickMultipole    & \ref{s:thickmult}   \\
  Fork             & \ref{s:fork}          &  Undulator         & \ref{s:undulator}   \\
  Instrument       & \ref{s:instrument}    &  UnionEle          & \ref{s:unionele}    \\
  Kicker           & \ref{s:kicker}        &  Wiggler           & \ref{s:wiggler}     \\
  \bottomrule
\end{tabular}
} 
\caption{Table of non-controller element types.}
\label{t:particle.classes}
\end{table}

\begin{table}[ht]
\centering
{\tt
\begin{tabular}{llll} \toprule
  {\it Element}  & {\it Section}     & {\it Element}  & {\it Section}    \\ \midrule
  Controller     & \ref{s:control}     &  Ramper        & \ref{s:ramper}   \\
  Girder         & \ref{s:girder}    &                &                  \\
 \\ \bottomrule
\end{tabular}
}
\caption{Table of controller elements.}
\label{t:control.classes}
\end{table}

\newpage

%-----------------------------------------------------------------
\section{BeamBeam}
\label{s:beambeam}


%-----------------------------------------------------------------
\section{BeginningEle}
\label{s:begin.ele}

A \vn{BeginningEle} element must be present as the first element of every tracking branch.
(\sref{s:branch.def}).

Element parameter groups associated with a \vn{BeginningEle} are:
\begin{example}
  LengthGroup          -> Length and s-position parameters.
  LordSlaveGroup       -> Element lord and slave status.
  StringGroup          -> Informational strings.
  ReferenceGroup       -> Reference energy and species.
  FloorPositionGroup   -> Global floor position and orientation.
  TrackingGroup        -> Default tracking settings.
\end{example}

\newpage

%-----------------------------------------------------------------
\section{Bend Element}
\label{s:bend}

A \vn{Bend} type element represents a dipole bend. 

%-----------------------------------------------------------------
\section{Collimator}
\label{s:collimator}


%-----------------------------------------------------------------
\section{CrabCavity}
\label{s:crab}


%-----------------------------------------------------------------
\section{Custom}
\label{s:custom}


%-----------------------------------------------------------------
\section{Drift}
\label{s:drift}


%-----------------------------------------------------------------
\section{EGun}
\label{s:egun}


%-----------------------------------------------------------------
\section{ElSeparator}
\label{s:elsep}


%-----------------------------------------------------------------
\section{EMField}
\label{s:emfield}


%-----------------------------------------------------------------
\section{Fiducial}
\label{s:fiducial}


%-----------------------------------------------------------------
\section{FloorShift}
\label{s:floorshift}


%-----------------------------------------------------------------
\section{foil}
\label{s:foil}


%-----------------------------------------------------------------
\section{Fork}
\label{s:fork}


%-----------------------------------------------------------------
\section{Instrument}
\label{s:instrument}


%-----------------------------------------------------------------
\section{Kicker}
\label{s:kicker}


%-----------------------------------------------------------------
\section{LCavity}
\label{s:lcav}


%-----------------------------------------------------------------
\section{Marker}
\label{s:marker}


%-----------------------------------------------------------------
\section{Match}
\label{s:match}


%-----------------------------------------------------------------
\section{NullEle}
\label{s:nullele}


%-----------------------------------------------------------------
\section{Octupole}
\label{s:oct}


%-----------------------------------------------------------------
\section{Patch}
\label{s:patch}


%-----------------------------------------------------------------
\section{Quadrupole}
\label{s:quad}


%-----------------------------------------------------------------
\section{RFCavity}
\label{s:rfcav}


%-----------------------------------------------------------------
\section{Sextupole}
\label{s:sex}


%-----------------------------------------------------------------
\section{Taylor}
\label{s:taylor}


%-----------------------------------------------------------------
\section{ThickMultipole}
\label{s:thickmult}


%-----------------------------------------------------------------
\section{Undulator}
\label{s:undulator}


%-----------------------------------------------------------------
\section{UnionEle}
\label{s:unionele}


%-----------------------------------------------------------------
\section{Wiggler}
\label{s:wiggler}


