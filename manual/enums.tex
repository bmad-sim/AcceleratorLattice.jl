\chapter{Enums and Holy Traits}
\label{c:enums}

Enums (\sref{s:enums}) and Holy traits (\sref{s:holy}) are used to define ``switches'' which are
variables whose value can be one of a set of named constants. 
A web search will provide documentation. 

The advantage of Holy traits is that they can be used with function dispatch. The disadvantage is
that the same Holy trait value name cannot be used with multiple groups.

%---------------------------------------------------------------------------------------------------
\section{Enums}
\label{s:enums}

\accellat uses the package \vn{EnumX.jl} to define enums (enumerated numbers).
Essentially what happens is that for each enum group there is a group name, For example \vn{BendType},
along with a set of values which, for \vn{BendType}, is \vn{SECTOR} and \vn{RECTANGULAR}. Values
are always referred to by their "full" name which in this example is \vn{BendType.SECTOR} and
\vn{BendType.RECTANGULAR}. Exception: \vn{BranchGeometry.CLOSED} and \vn{BranchGeometry.OPEN} are
used often enough so that the constants \vn{OPEN} and \vn{CLOSED} are defined.

The group name followed by a \vn{.T} suffix denotes the enum type.
For example:
\begin{example}
  struct ApertureGroup <: EleParameterGroup
    aperture_type::ApertureShape.T = ApertureShape.ELLIPTICAL
    aperture_at::BodyLoc.T = BodyLoc.ENTRANCE_END
    ...
\end{example}

The \vn{enum} function is used to convert a list into an enum group and export the names.
The \vn{enum} function also overloads \vn{Base.string} so that something like \vn{string(Lord.NOT)} 
will return \vn{"Lord.NOT"} instead of just \vn{"NOT"} (an issue with the EnumX.jl package). 
See the documentation for \vn{enum} for more details.

The \vn{enum_add} function is used to add values to an existing enum group. See the documentation for
\vn{enum_add} for more details. This function is used with code extensions to customize \accellat.


List of enum groups defined in \accellat:
\begin{description}[leftmargin=1em]
%
\item[BendType] --- Type of Bend element magnet. \Newline
\vskip -2ex
\begin{tabular}{ll}
  .SECTOR      & --- Sector shape \\
  .RECTANGULAR & --- Rectangular shape \\
\end{tabular} 
\hfill \break \vskip -1.2ex
There is no difference between \vn{SECTOR} and \vn{RECTANGULAR} bends except in the case where
the bend curvature is varied. In this case, for a \vn{SECTOR} bend the face angles \vn{e1} and
\vn{e2} are held constant and \vn{e1_rect} and \vn{e2_rect} are varied to keep ...
\vskip 3ex
%
\item[BodyLoc] --- Longitudinal location with respect to element's body coordinates.\Newline
\vskip -2ex
\begin{tabular}{ll}
  .ENTRANCE_END & --- Body entrance end \\
  .CENTER       & --- Element center \\
  .EXIT_END     & --- Body exit end \\
  .BOTH_ENDS    & --- Both ends \\
  .NOWHERE      & --- No location \\
  .EVERYWHERE   & --- Everywhere \\
\end{tabular}
\hfill \break \vskip -1.2ex
%
\item[BranchGeometry] --- Geometry of a lattice branch\Newline
\vspace*{-5pt}
\begin{tabular}{ll}
  .OPEN    & --- Open geometry like a Linac. \\
  .CLOSED  & --- Closed geometry like a storage ring.
\end{tabular}
\hfill \break \vskip -1.2ex
%
\item[Cavity] --- Type of RF cavity. \Newline
\vspace*{-5pt}
\begin{tabular}{ll}
  .STANDING_WAVE   & --- Standing wave cavity \\
  .TRAVELING_WAVE  & --- Traveling wave cavity \\
\end{tabular}
\hfill \break \vskip -1.2ex
%
\item[Lord] --- Type of lord an element is \Newline
\vspace*{-5pt}
\begin{tabular}{ll}
  .NOT       & --- Not a lord \\
  .SUPER     & --- Super lord \\
  .MULTIPASS & --- Multipass lord \\
  .GOVERNOR  & --- Girder and other "minor" lords \\ 
\end{tabular}
\hfill \break \vskip -1.2ex
%
\item[Slave] --- Type of slave an element is \Newline
\vspace*{-5pt}
\begin{tabular}{ll}
  .NOT       & --- Not a slave \\
  .SUPER     & --- Super slave \\
  .MULTIPASS & --- Multipass slave \\
\end{tabular}
\hfill \break \vskip -1.2ex
%
\item[Loc] --- Longitudinal location with respect to element's machine coordinates. \Newline
\vspace*{-5pt}
\begin{tabular}{ll}
  .UPSTREAM_END   & --- Upstream element end\\
  .CENTER         & --- center of element \\
  .INSIDE         & --- Somewhere inside \\
  .DOWNSTREAM_END & --- Downstream element end \\
\end{tabular}
\hfill \break \vskip -1.2ex
%
\item[Select] --- Specifies where to select if there is a choice of elements or positions. \Newline
\vspace*{-5pt}
\begin{tabular}{ll}
  .UPSTREAM   & --- Select upstream \\
  .DOWNSTREAM & --- Select downstream \\
\end{tabular}
\hfill \break \vskip -1.2ex
%
\item[ExactMultipoles] --- How multipoles are handled in a Bend element \Newline
\vspace*{-5pt}
\begin{tabular}{ll}
  .OFF               & --- Bend curvature not taken into account. \\
  .HORIZONTALLY_PURE & --- Coefficients correspond to horizontally pure multipoles. \\
  .VERTICALLY_PURE   & --- Coefficients correspond to vertically pure multipoles. \\
\end{tabular}
\hfill \break \vskip -1.2ex
%
\item[FiducialPt] Fiducial point location with respect to element's machine coordinates. \Newline
\vspace*{-5pt}
\begin{tabular}{ll}
  .ENTRANCE_END & --- Entrance end of element \\
  .CENTER       & --- Center of element \\
  .EXIT_END     & --- Exit end of element \\
  .NONE         & --- No point chosen \\
\end{tabular}
%
\end{description}

%---------------------------------------------------------------------------------------------------
\section{Holy Traits}
\label{s:holy}

\vn{Holy traits} (named after Tim Holy) are a design pattern in Julia that that behave similarly
to \vn{enums} (\sref{s:enum}). A Holy trait group consists of an abstract type with a set of values
(traits) which are structs that inherit from the abstract type.

There is a convenience function \vn{holy_traits} which will define a traits group, export the names,
and create a docstring for the group. Values can be added to an existing group by defining a 
new struct that inherits from the group abstract type.

\begin{description}
%
\item[ApertureShape] --- The shape of an aperture.\Newline 
\vspace*{-5pt}
\begin{tabular}{ll}
  RECTANGULAR & --- Rectangular shape \\
  ELLIPTICAL  & --- Elliptical shape \\
\end{tabular}
%
\end{description}


