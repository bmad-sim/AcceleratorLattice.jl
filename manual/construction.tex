\chapter{Constructing Lattices}
\label{c:construct-lat}

%---------------------------------------------------------------------------------------------------
\section{Defining a Lattice Element}
\label{s:ele.def}

Chapter~\sref{c:ele} gives a list of lattice elements defined by \accellat. 
Lattice elements are instantiated from structs which inherit from the abstract type \vn{Lat}.

Elements are defined using the \vn{@ele} macro. The general syntax is:
\begin{example}
  @ele eleName = eleType(param1 = val1, param2 = val2, ...)
\end{example}
where \vn{eleName} is the name of the element, \vn{eleType} is the type of element, \vn{param1}, \vn{param2},
etc. are parameter names and \vn{val1}, \vn{val2}, etc. are the parameter values.
Example:
\begin{example}
  @ele qf = Quadrupole(L = 0.6, K1 = 0.370)
\end{example}
The \vn{@ele} macro will construct a \julia variable with the name \vn{eleName}. Additionally the element
that this variable references will also hold \vn{eleName} as the name of the element. So with this
example, \vn{qf.name} will be the string \vn{"qf"}. If multiple elements are being defined, a single
\vn{@eles} macro can be used instead of multiple \vn{@ele} macros. Example:
\begin{example}
  @eles begin
    s1 = Sextupole(L = ...)
    b2 = Bend(...)
    ...
  end
\end{example}

%---------------------------------------------------------------------------------------------------
\section{Element Parameters}
\label{s:ele.type}

Generally, element parameters are grouped into ``element parameter groups'' structs which inherit
from the abstract type \vn{EleParameterGroup}.
For example, the \vn{LengthGroup} holds the length and s-positions of the element and is defined by:
\begin{example}
  @kwdef struct LengthGroup <: EleParameterGroup
    L::Number = 0
    s::Number = 0
    s_downstream::Number = 0
    orientation::Int = 1
  end
\end{example}
The \vn{\@kwdef} macro automatically defines a keyword-based constructor for \vn{LengthGroup}. 
To see a list of all element parameter groups use the \vn{subtypes(EleParamterGroup)} command.
To see the components of a given group use the \vn{fieldnames} function. For information on
a given element parameter use the \vn{info(::Symbol)} function. For example:
\begin{example}
  julia> info(:s_downstream)
    User name:       s_downstream
    Stored in:       LengthGroup.s_downstream
    Parameter type:  Number
    Units:           m
    Description:     Longitudinal s-position at the downstream end.
\end{example}
Notice that the argument to the \vn{info} function is the symbol associated with the parameter.
the ``User name'' is the name used when setting the parameter. This is discussed below. For most
parameters, the User name and the name of the corresponding component in the element parameter
group is the same. But there are exceptions. For example:
\begin{example}
  julia> info(:theta)
    User name:       theta_floor
    Stored in:       FloorPositionGroup.theta
    Parameter type:  Number
    Units:           rad
    Description:     Element floor theta angle orientation
\end{example}

%---------------------------------------------------------------------------------------------------
\section{Anatomy of an Element}
\label{s:ele.anatomy}

The structs for all elements types contain exactly one component which is a Dict called \vn{pdict}
(short for ``parameter dict'') which is of type \vn{Dict\{Symbol,Any\}}.

Element parameter groups~\sref{s:ele.type} are stored in an element's \vn{pdict}, with the key 
being the symbol associated with the group. For example, an element \vn{ele} that has a
\vn{LengthGroup} will store this group at \vn{ele.pdict[:LengthGroup]} and the length component \vn{L}
of this group is accessed by \vn{ele.pdict[:LengthGroup].L}.

The \vn{Base.setproperty} and \vn{Base.getproperty} functions, which get called when the dot
selection operator is used, have been overloaded for elements so that \vn{ele.L} will get mapped to 
\vn{ele.pdict[:LengthGroup].L}. Thus the following two statements both set the \vn{s_downstream}
parameter of an element named \vn{q1}:
\begin{example}
  q1.pdict[:Length_group].s_downstream = q1.pdict[:Length_group].s + 
                                                     q1.pdict[:Length_group].L
  q1.s_downstream = q1.s + q1.L
\end{example}
These two statements are not equivalent however. The difference is that when \vn{s_downstream}
is set using \vn{q1.s_downstream}, the set is recorded by adding an entry to a 
Dict in the element at \vn{ele.pdict[:changed]} where \vn{ele} is the element whose parameter
was changed. The key of the entry will be the symbol (\vn{:s_downstream} in the present example)
associated with the parameter and the value will be the old value of the parameter. When
the \vn{bookkeeper(::Lat)} function is called, the bookkeeping code will use the
entries in \vn{ele.pdict[:changed]} to limit the bookkeeping to what is necessary and thus
minimize computing time. Knowing what has been changed is also important in resolving what
parameters need to be changed. 
For example, if the bend \vn{angle} of a bend is changed, the bookkeeping code will set the 
bending strength \vn{g} using the equation \vn{g} = \vn{angle} / \vn{L}. If, instead, if
\vn{g} is changed, the bookkeeping code will set \vn{angle} appropriately. 



%---------------------------------------------------------------------------------------------------
\section{Bookkeeping and Dependent Element Parameters}
\label{s:param.depend}

After lattice parameters are changed, the function \vn{bookkeeper(::Lat)} needs to be called
so that dependent parameters can be updated. 
Since bookkeeping can take a significant amount of time if bookkeeping is done every time
a change to the lattice is made, and since there is no good way to tell when bookkeeping should
be done, After lattice expansion, \vn{bookkeeper(::Lat)} is never called directly by \accellat functions and needs to be called by the User when appropriate (generally before tracking or
other computations are done).

Broadly there are two types of dependent parameters: intra-element dependent parameters where
the changed parameters and the dependent parameters are all within the same element and
cascading dependent parameters where changes to one element cause changes to parameters of 
elements downstream.

The cascading dependencies are:
\begin{description}
%
\item [s-position dependency:]
Changes to an elements length \vn{L} or changes to the beginning element's \vn{s} parameter will
result in the s-positions of all downstream elements changing.
%
\item [Reference energy dependency:] Changes to the be beginning element's reference energy (or
equivilantly the referece momentum), or changes to the \vn{voltage} of an \vn{LCavity} element
will result in the reference energy of all downstream elements changing.
%
\item[Global position dependency:]
The position of a lattice element in the global coordinate system (\sref{s:global}) is affected
by a) the lengths of all upstream elements, b) the bend angles of all upstream elements, and c)
the position in global coordinates of the beginning element.
\end{description}


%---------------------------------------------------------------------------------------------------
\section{Defining a Lattice Element Type}
\label{s:ele.type}

All lattice element types like \vn{Quadrupole}, \vn{Marker}, etc. are subtypes of the abstract type
\vn{Ele}. To construct a new type, use the \vn{\@construct_ele_type} macro. Example:
\begin{example}
  @construct_ele_type MyEleType
\end{example}
