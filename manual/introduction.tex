\chapter{Introduction, Overview, and Concepts}

This chapter is an overview of, and an introduction to, the \accellat package which is part of the
greater \bmadjl ecosystem of toolkits  and programs for accelerator simulations. With \accellat,
lattices, which can be used to describe such
things as LINACs, storage rings, X-ray beam lines,
can be constructed and manipulated. Tracking and lattice analysis (for example, calculating
closed orbits and Twiss functions) is left to other packages in the \bmadjl ecosystem.

There are three main sources of documentation. One source is this manual which is written in LaTex
and distributed in pdf format. A second source an introduction and overview guide written in Markdown.
Finally, taking advantage of the 

%---------------------------------------------------------------------------------------------------
\section{Acknowledgements}

It is my pleasure to express appreciation to people who have contributed to this effort, and without
whom, \bmadjl would only be a shadow of what it is today: 

\'Etienne Forest (aka Patrice Nishikawa),
Matthew Signorelli,
Alexander Coxe,
Oleksii Beznosov,
Ryan Foussel,
Auralee Edelen,
Chris Mayes,
Georg Hoffstaetter,
Juan Pablo Gonzalez-Aguilera,
Scott Berg,
Dan Abell,
Laurent Deniau, and
Hugo Slepicka

%---------------------------------------------------------------------------
\section{}


The \julia language itself is used as the basis for constructing lattices with \accellat. 
Other simulation programs
have similarly utilized the underlying programming language for constructing 
lattices\cite{Appleby:Merlin2020,Iadarola:Xsuite2023}. This is in marked contrast to many accelerator
simulation programs such programs as MAD\cite{Grote:MAD1989}, Elegant\cite{Borland:Elegant2000}, and the 
\bmad toolkit\cite{Sagan:Bmad2006}. 



%---------------------------------------------------------------------------
\section{Using AcceleratorLattice.jl}

\accellat is hosted on GitHub. The official repository is at
\begin{example}
  github.com/bmad-sim/AcceleratorLattice.jl
\end{example}

A \vn{using} statement must be given before using \accellat
\begin{example}
  using AcceleratorLattice
\end{example}

%---------------------------------------------------------------------------
\section{Lattice Elements}
\label{s:element.def}

The basic building block used to describe an accelerator is the lattice \vn{element}. An
element can be a physical thing that particles travel ``through'' like a bending magnet, a
quadrupole or a Bragg crystal, or something like a \vn{marker} element (\sref{s:mark}) that is used
to mark a particular point in the machine.  Besides physical elements, there are \vn{controller}
elements that can be used for parameter control of other elements.

Lattice elements are Julia \vn{struct}s that inherit from the abstract type \vn{Lat}.

Chapter~\sref{c:elements} lists the complete set of different element types that \bmad knows about.

In a lattice \vn{branch} (\sref{s:branch.def}), each element in the
ordered array of elements are assigned an \vn{element index}
starting from one. The first element in a branch array
is called \vn{beginning_ele} (\sref{s:begin.ele}).
This element is always included in every \vn{branch} \sref{s:branch.def} and is used as a
marker for the beginning of the \vn{branch}.  Additionally, every branch will have a final
marker element (\sref{s:mark}) named \vn{end_ele}.

%---------------------------------------------------------------------------
\section{Lattice Branches}
\label{s:branch.def}

The next level up from an \vn{element} is the \vn{branch}. A 
\vn{branch} contains an ordered sequence of lattice elements that a particle will travel through. A
branch can represent a LINAC, X-Ray beam line, storage ring or anything else that can be represented
as a simple ordered list of elements.

Chapter~\sref{c:sequence} shows how a \vn{branch} can be defined using \vn{line}s.

Branches can be interconnected using \vn{fork} elements (\sref{s:fork}). This
is used to simulate forking beam lines such as a connections to a transfer line, dump line, or an
X-ray beam line. A \vn{branch} from which other \vn{branches} fork but is not forked to by any
other \vn{branch} is called a \vn{root} branch. A branch that is forked to by some other branch
is called a \vn{downstream} branch.

There are two types of \vn{branches}: \vn{LordBranches} and \vn{TrackingBranches}, Branches whose \vn{Branch.type} are set to \vn{LordBranch}


%---------------------------------------------------------------------------
\section{Lattice}
\label{s:lattice.def}

A \vn{lattice} (\sref{s:lattice.def}), has an array of \vn{branches}. 
Each \vn{branch} in this array
has a name an is assigned an index starting from one. 
Branches are named after the line that defines the \vn{branch}.

A \vn{lattice} contains an array of \vn{branches} that can be interconnected 
together to describe an entire machine
complex. A \vn{lattice} can include such things as transfer lines, dump lines, x-ray beam lines,
colliding beam storage rings, etc. All of which can be connected together to form a coherent whole. 
In addition, a lattice may contain \vn{controller elements} (Table~\ref{t:control.classes}) 
which can
simulate such things as magnet power supplies and support structures like a girder supporting
a section of magnets or an optical table supporting photonic elements.

Branches can be interconnected using \vn{fork} and \vn{photon_fork} elements (\sref{s:fork}). This
is used to simulate forking beam lines such as a connections to a transfer line, dump line, or an
X-ray beam line. The \vn{branch} from which other \vn{branches} fork but is not forked to by any
other \vn{branch} is called a \vn{root} branch.

A lattice may contain multiple \vn{root} \vn{branches}. For example, a pair of intersecting storage
rings will generally have two \vn{root} branches, one for each ring. The \vn{use} statement
(\sref{s:use}) in a lattice file will list the \vn{root} \vn{branches} of a lattice. To connect
together lattice elements that are physically shared between branches, for example, the interaction
region in colliding beam machines, \vn{multipass} lines (\sref{s:multipass}) can be used.


